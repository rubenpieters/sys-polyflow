% listings
\newcommand{\ts}[1]{\lstinline+#1+}

% terms
\newcommand{\abs}[4]{\lambda (#1 : #2) . (#3 : #4)}
\newcommand{\babs}[2]{\lambda #1 . #2}
\newcommand{\app}[2]{#1\,#2}
\newcommand{\tabs}[3]{\Lambda #1 <: #2 . #3}
\newcommand{\tapp}[2]{#1\,[#2]}
\newcommand{\ite}[3]{\texttt{if}\ #1\ \texttt{then}\ #2\ \texttt{else}\ #3}
\newcommand{\true}[0]{\texttt{true}}
\newcommand{\false}[0]{\texttt{false}}
\newcommand{\eq}[2]{#1\,\texttt{==}\,#2}
\newcommand{\tand}[2]{#1\,\texttt{\&\&}\,#2}
\newcommand{\tor}[2]{#1\,\texttt{||}\,#2}

% types
\newcommand{\fun}[2]{#1 \rightarrow #2}
\newcommand{\tfun}[3]{\forall #1 <: #2 . #3}
\newcommand{\ttop}[0]{Top}
\newcommand{\tbool}[0]{Bool}
\newcommand{\ttrue}[0]{True}
\newcommand{\tfalse}[0]{False}
\newcommand{\union}[2]{#1 \lor #2}
\newcommand{\intersect}[2]{#1 \land #2}
\newcommand{\tmap}[3]{\{~\texttt{true}:~#1,~\texttt{false}:~#2~\}[#3]}

% typing relation
\newcommand{\hastype}[3]{#1 \vdash #2 : #3}
\newcommand{\hassubtype}[3]{#1 \vdash #2 <: #3}
\newcommand{\nothassubtype}[3]{#1\ \cancel{\vdash}\ #2 <: #3}
\newcommand{\isatleast}[3]{#1 \vdash #2 \subset #3}
\newcommand{\notatleast}[3]{#1\ \cancel{\vdash}\ #2 \subset #3}

% bidirectional typing
\newcommand{\checks}[3]{#1\ \vdash #2\ \textcolor{blue}{\Leftarrow}\ #3}
\newcommand{\infers}[3]{#1\ \vdash #2\ \textcolor{red}{\Rightarrow}\ #3}
\newcommand{\ann}[2]{(#1~:~#2)}

% context
\newcommand{\typeincontext}[3]{#1 : #2 \in #3}
\newcommand{\subtypeincontext}[3]{#1 <: #2 \in #3}
\newcommand{\supertypeincontext}[3]{#1 :> #2 \in #3}
\newcommand{\emptycontext}[0]{\emptyset}
\newcommand{\extendwtype}[3]{#1 , #2 : #3}
\newcommand{\extendwsubtype}[3]{#1 , #2 <: #3}
\newcommand{\extendwsupertype}[3]{#1 , #2 :> #3}
\newcommand{\extendwatleast}[3]{#1 , #2\,\texttt{<must-atleast-be>}\,#3}
\newcommand{\atleast}[3]{#1 \vdash #2\,\texttt{<must-atleast-be>}\,#3}
\newcommand{\extendwcannot}[3]{#1 , #2\,\texttt{<cannot-be-exactly>}\,#3}
\newcommand{\cannot}[3]{#1 \vdash #2\texttt{<cannot-be-exactly>}\,#3}
\newcommand{\cstop}[2]{#1 \vdash \texttt{stop}\,#2}

% type evaluation
\newcommand{\teval}[3]{#1 \vdash #2 \xrightarrow[\texttt{m}]{TE} #3}
\newcommand{\tevalx}[3]{#1 \vdash #2 \xrightarrow[\texttt{m}]{TE}^* #3}
\newcommand{\tevalr}[3]{#1 \vdash #2 \xrightarrow[r]{TE} #3}
\newcommand{\tevalrx}[3]{#1 \vdash #2 \xrightarrow[r]{TE}^* #3}
\newcommand{\tevalw}[3]{#1 \vdash #2 \xrightarrow[w]{TE} #3}
\newcommand{\tevalwx}[3]{#1 \vdash #2 \xrightarrow[w]{TE}^* #3}
\newcommand{\teq}[3]{#1 \vdash #2 =_{TE} #3}

% term evaluation
\newcommand{\eval}[2]{#1 \rightarrow #2}
\newcommand{\concrete}[1]{\mathit{concrete}\,#1}

% substitution
\newcommand{\subst}[3]{#1{}[#2 \mapsto #3]}
\newcommand{\tsubst}[3]{#1{}[#2 \mapsto #3]}

% grammar
\newcommand{\alt}{\ |\ }

% condition info
\newcommand{\ci}[3]{#1 \vdash #2 \rightsquigarrow #3}

% context operation
\newcommand{\coia}[3]{#1 \sqcap #2 \rightarrow #3}
\newcommand{\coi}[2]{#1 \sqcap #2}
\newcommand{\coua}[3]{#1 \sqcup #2 \rightarrow #3}
\newcommand{\cou}[2]{#1 \sqcup #2}
\newcommand{\cona}[2]{\lnot #1 \rightarrow #2}
\newcommand{\con}[1]{\lnot #1}
