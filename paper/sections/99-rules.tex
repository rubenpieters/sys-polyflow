\section{Rules}

\subsection{Typing Relation}

\begin{mathpar}

\inferrule*[right=T-True]
           {  }
           { \hastype{\Gamma}{\true}{\ttrue} }

\inferrule*[right=T-False]
           {  }
           { \hastype{\Gamma}{\false}{\tfalse} }

\inferrule*[right=T-Var]
           { \typeincontext{x}{T}{\Gamma} }
           { \hastype{\Gamma}{x}{T} }

\inferrule*[right=T-Abs]
           { \hastype{\extendwtype{\Gamma}{x}{A}}{t}{B} }
           { \hastype{\Gamma}{(\abs{x}{A}{t}{B})}{(\fun{A}{B})} }

\inferrule*[right=T-App]
           { \hastype{\Gamma}{t_1}{\fun{A}{B}} \\ \hastype{\Gamma}{t_2}{A} }
           { \hastype{\Gamma}{\app{t_1}{t_2}}{B} }

\inferrule*[right=T-TAbs]
           { \hastype{\extendwsubtype{\Gamma}{X}{U}}{t}{T} }
           { \hastype{\Gamma}{(\tabs{X}{U}{t})}{(\tfun{X}{U}{T})} }

\inferrule*[right=T-TApp]
           { \hastype{\Gamma}{t}{\tfun{X}{U}{T}} \\ \hassubtype{\Gamma}{S}{U} }
           { \hastype{\Gamma}{\tapp{t}{S}}{\subst{T}{X}{S}} }

\inferrule*[right=T-Sub]
           { \hastype{\Gamma}{t}{S} \\ \hassubtype{\Gamma}{S}{T} }
           { \hastype{\Gamma}{t}{T} }

\inferrule*[right=T-Eq]
           { \hastype{\Gamma}{t_1}{T} \\ \hastype{\Gamma}{t_2}{T} }
           { \hastype{\Gamma}{\eq{t_1}{t_2}}{\tbool} }

\inferrule*[right=T-If]
           { \hastype{\Gamma}{t_1}{\tbool} \\ \hastype{\Gamma}{t_2}{A} \\ \hastype{\Gamma}{t_3}{B} }
           { \hastype{\Gamma}{\ite{t_1}{t_2}{t_3}}{\union{A}{B}} }

% NOTE: in flow-typing system with intersection/negation types, it becomes Gamma [ x |-> Gamma(x) & T ] in 'x is T' branch, and Gamme [ x |-> Gamma(x) & not(T) ] in else branch
% for example in Sound and Complete Flow Typing with Unions, Intersections and Negations           

\inferrule*[right=T-IfTrue]
           { \hastype{\Gamma}{x}{X} \\ \hastype{\extendwsubtype{\tsubst{\Gamma}{x}{True}}{\ttrue}{X}}{t_2}{A} \\ \hastype{\extendwsubtype{\tsubst{\Gamma}{x}{False}}{\tfalse}{X}}{t_3}{B} }
           { \hastype{\Gamma}{\ite{\eq{x}{\true}}{t_2}{t_3}}{\union{A}{B}} }

\end{mathpar}

\subsection{Sub-Typing Relation 1: Sub-Typing Approach}

\begin{mathpar}

\inferrule*[right=S-Refl]
           {  }
           { \hassubtype{\Gamma}{S}{S} }

\inferrule*[right=S-Trans]
           { \hassubtype{\Gamma}{S}{U} \\ \hassubtype{\Gamma}{U}{T} }
           { \hassubtype{\Gamma}{S}{T} }

           \inferrule*[right=S-Top]
           {  }
           { \hassubtype{\Gamma}{S}{Top} }

\inferrule*[right=S-TVarSub]
           { \subtypeincontext{X}{T}{\Gamma} }
           { \hassubtype{\Gamma}{X}{T} }

\inferrule*[right=S-TVarSup]
           { \supertypeincontext{X}{T}{\Gamma} }
           { \hassubtype{\Gamma}{T}{X} }

\inferrule*[right=S-Arrow]
           { \hassubtype{\Gamma}{T_1}{S_1} \\ \hassubtype{\Gamma}{S_2}{T_2} }
           { \hassubtype{\Gamma}{(\fun{S_1}{S_2})}{(\fun{T_1}{T_2})} }

\inferrule*[right=S-All]
           { \hassubtype{\extendwsubtype{\Gamma}{X}{U}}{S}{T}}
           { \hassubtype{\Gamma}{(\tfun{X}{U}{S})}{(\tfun{X}{U}{T})} }

\inferrule*[right=S-UnionL]
           { \hassubtype{\Gamma}{T}{L} }
           { \hassubtype{\Gamma}{T}{\union{L}{R}} }

\inferrule*[right=S-UnionR]
           { \hassubtype{\Gamma}{T}{R} }
           { \hassubtype{\Gamma}{T}{\union{L}{R}} }

\inferrule*[right=S-UnionM]
           { \hassubtype{\Gamma}{L}{T} \\ \hassubtype{\Gamma}{R}{T} }
           { \hassubtype{\Gamma}{\union{L}{R}}{T} }

\inferrule*[right=S-MapTrueL]
           {  }
           { \hassubtype{\Gamma}{\tmap{T_t}{T_f}{\ttrue}}{T_t} }

\inferrule*[right=S-MapTrueR]
           {  }
           { \hassubtype{\Gamma}{T_t}{\tmap{T_t}{T_f}{\ttrue}} }

\inferrule*[right=S-MapFalseL]
           {  }
           { \hassubtype{\Gamma}{\tmap{T_t}{T_f}{\tfalse}}{T_f} }

\inferrule*[right=S-MapFalseR]
           {  }
           { \hassubtype{\Gamma}{T_f}{\tmap{T_t}{T_f}{\tfalse}} }

\inferrule*[right=S-MapBool]
           {  }
           { \hassubtype{\Gamma}{\tmap{T_t}{T_f}{\tbool}}{\union{T_t}{T_f}} }

\inferrule*[right=S-Map]
           { \hassubtype{\Gamma}{S}{T} }
           { \hassubtype{\Gamma}{\tmap{T_t}{T_f}{S}}{\tmap{T_t}{T_f}{T}} }

\end{mathpar}

Type $\tbool$ is defined as $\union{\ttrue}{\tfalse}$.

\subsection{Sub-Typing Relation 2: Type Evaluation Approach}

\begin{mathpar}

\inferrule*[right=S-Refl]
           {  }
           { \hassubtype{\Gamma}{S}{S} }

\inferrule*[right=S-Trans]
           { \hassubtype{\Gamma}{S}{U} \\ \hassubtype{\Gamma}{U}{T} }
           { \hassubtype{\Gamma}{S}{T} }

           \inferrule*[right=S-Top]
           {  }
           { \hassubtype{\Gamma}{S}{Top} }

\inferrule*[right=S-TVarSub]
           { \subtypeincontext{X}{T}{\Gamma} }
           { \hassubtype{\Gamma}{X}{T} }

\inferrule*[right=S-TVarSup]
           { \supertypeincontext{X}{T}{\Gamma} }
           { \hassubtype{\Gamma}{T}{X} }

\inferrule*[right=S-Arrow]
           { \hassubtype{\Gamma}{T_1}{S_1} \\ \hassubtype{\Gamma}{S_2}{T_2} }
           { \hassubtype{\Gamma}{(\fun{S_1}{S_2})}{(\fun{T_1}{T_2})} }

\inferrule*[right=S-All]
           { \hassubtype{\extendwsubtype{\Gamma}{X}{U}}{S}{T}}
           { \hassubtype{\Gamma}{(\tfun{X}{U}{S})}{(\tfun{X}{U}{T})} }

\inferrule*[right=S-UnionL]
           { \hassubtype{\Gamma}{T}{L} }
           { \hassubtype{\Gamma}{T}{\tbool} }

\inferrule*[right=S-UnionR]
           { \hassubtype{\Gamma}{T}{R} }
           { \hassubtype{\Gamma}{T}{\union{L}{R}} }

\inferrule*[right=S-UnionM]
           { \hassubtype{\Gamma}{L}{T} \\ \hassubtype{\Gamma}{R}{T} }
           { \hassubtype{\Gamma}{\union{L}{R}}{T} }

\inferrule*[right=S-TEval]
           { \tevalr{\Gamma}{T_1}{T_2} \\ \tevalw{\Gamma}{T_2}{T_1} }
           { \hassubtype{\Gamma}{T_1}{T_2} }

\inferrule*[right=S-Map]
           { \hassubtype{\Gamma}{S}{T} }
           { \hassubtype{\Gamma}{\tmap{T_t}{T_f}{S}}{\tmap{T_t}{T_f}{T}} }

\end{mathpar}

\subsection{Type Evaluation}

\begin{mathpar}
          
\inferrule*[right=TE-MapTrue]
           {  }
           { \teval{\Gamma}{\tmap{T_t}{T_f}{\ttrue}}{T_t} }

\inferrule*[right=TE-MapFalse]
           {  }
           { \teval{\Gamma}{\tmap{T_t}{T_f}{\tfalse}}{T_f} }

\inferrule*[right=TE-MapBoolRead]
           {  }
           { \tevalr{\Gamma}{\tmap{T_t}{T_f}{\tbool}}{\union{T_t}{T_f}} }

\inferrule*[right=TE-MapBoolWrite]
           {  }
           { \tevalw{\Gamma}{\tmap{T_t}{T_f}{\tbool}}{\intersect{T_t}{T_f}} }


\end{mathpar}
