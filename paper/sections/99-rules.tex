\section{Rules}

\subsection{Typing Relation}

\begin{mathpar}

\inferrule*[right=T-True]
           {  }
           { \hastype{\Gamma}{\true}{\ttrue} }

\inferrule*[right=T-False]
           {  }
           { \hastype{\Gamma}{\false}{\tfalse} }

\inferrule*[right=T-Var]
           { \typeincontext{x}{T}{\Gamma} }
           { \hastype{\Gamma}{x}{T} }

\inferrule*[right=T-Abs]
           { \hastype{\extendwtype{\Gamma}{x}{A}}{t}{B} }
           { \hastype{\Gamma}{(\abs{x}{A}{t}{B})}{(\fun{A}{B})} }

\inferrule*[right=T-App]
           { \hastype{\Gamma}{t_1}{\fun{A}{B}} \\ \hastype{\Gamma}{t_2}{A} }
           { \hastype{\Gamma}{\app{t_1}{t_2}}{B} }

\inferrule*[right=T-TAbs]
           { \hastype{\extendwsubtype{\Gamma}{X}{U}}{t}{T} }
           { \hastype{\Gamma}{(\tabs{X}{U}{t})}{(\tfun{X}{U}{T})} }

\inferrule*[right=T-TApp]
           { \hastype{\Gamma}{t}{\tfun{X}{U}{T}} \\ \hassubtype{\Gamma}{S}{U} }
           { \hastype{\Gamma}{\tapp{t}{S}}{\subst{T}{X}{S}} }

\inferrule*[right=T-Sub]
           { \hastype{\Gamma}{t}{S} \\ \hassubtype{\Gamma}{S}{T} }
           { \hastype{\Gamma}{t}{T} }

\inferrule*[right=T-Eq]
           { \hastype{\Gamma}{t_1}{T} \\ \hastype{\Gamma}{t_2}{T} }
           { \hastype{\Gamma}{\eq{t_1}{t_2}}{\tbool} }

\inferrule*[right=T-If]
           { \hastype{\Gamma}{t_1}{\tbool} \\ \hastype{\Gamma}{t_2}{A} \\ \hastype{\Gamma}{t_3}{B} }
           { \hastype{\Gamma}{\ite{t_1}{t_2}{t_3}}{\union{A}{B}} }

% NOTE: in flow-typing system with intersection/negation types, it becomes Gamma [ x |-> Gamma(x) & T ] in 'x is T' branch, and Gamma [ x |-> Gamma(x) & not(T) ] in else branch
% for example in Sound and Complete Flow Typing with Unions, Intersections and Negations           

\inferrule*[right=T-IfTrue]
           { \hastype{\Gamma}{x}{X} \\ \hastype{\extendwsubtype{\tsubst{\Gamma}{x}{\true}}{\ttrue}{X}}{t_2}{A} \\ \hastype{\extendwsubtype{\tsubst{\Gamma}{x}{\tfalse}}{\tfalse}{X}}{t_3}{B} }
           { \hastype{\Gamma}{\ite{\eq{x}{\true}}{t_2}{t_3}}{\union{A}{B}} }

\end{mathpar}

\subsection{Sub-Typing Relation 1: Sub-Typing Approach}

\begin{mathpar}

\inferrule*[right=S-Refl]
           {  }
           { \hassubtype{\Gamma}{S}{S} }

\inferrule*[right=S-Trans]
           { \hassubtype{\Gamma}{S}{U} \\ \hassubtype{\Gamma}{U}{T} }
           { \hassubtype{\Gamma}{S}{T} }

           \inferrule*[right=S-Top]
           {  }
           { \hassubtype{\Gamma}{S}{\ttop} }

\inferrule*[right=S-TVarSub]
           { \subtypeincontext{X}{T}{\Gamma} }
           { \hassubtype{\Gamma}{X}{T} }

\inferrule*[right=S-TVarSup]
           { \supertypeincontext{X}{T}{\Gamma} }
           { \hassubtype{\Gamma}{T}{X} }

\inferrule*[right=S-Arrow]
           { \hassubtype{\Gamma}{T_1}{S_1} \\ \hassubtype{\Gamma}{S_2}{T_2} }
           { \hassubtype{\Gamma}{(\fun{S_1}{S_2})}{(\fun{T_1}{T_2})} }

\inferrule*[right=S-All]
           { \hassubtype{\extendwsubtype{\Gamma}{X}{U}}{S}{T}}
           { \hassubtype{\Gamma}{(\tfun{X}{U}{S})}{(\tfun{X}{U}{T})} }

\inferrule*[right=S-UnionL]
           { \hassubtype{\Gamma}{T}{L} }
           { \hassubtype{\Gamma}{T}{\union{L}{R}} }

\inferrule*[right=S-UnionR]
           { \hassubtype{\Gamma}{T}{R} }
           { \hassubtype{\Gamma}{T}{\union{L}{R}} }

\inferrule*[right=S-UnionM]
           { \hassubtype{\Gamma}{L}{T} \\ \hassubtype{\Gamma}{R}{T} }
           { \hassubtype{\Gamma}{\union{L}{R}}{T} }

\inferrule*[right=S-MapTrueL]
           {  }
           { \hassubtype{\Gamma}{\tmap{T_t}{T_f}{\ttrue}}{T_t} }

\inferrule*[right=S-MapTrueR]
           {  }
           { \hassubtype{\Gamma}{T_t}{\tmap{T_t}{T_f}{\ttrue}} }

\inferrule*[right=S-MapFalseL]
           {  }
           { \hassubtype{\Gamma}{\tmap{T_t}{T_f}{\tfalse}}{T_f} }

\inferrule*[right=S-MapFalseR]
           {  }
           { \hassubtype{\Gamma}{T_f}{\tmap{T_t}{T_f}{\tfalse}} }

\inferrule*[right=S-MapBoolL]
           {  }
           { \hassubtype{\Gamma}{\tmap{T_t}{T_f}{\tbool}}{\union{T_t}{T_f}} }

\inferrule*[right=(S-MapBoolR)]
           {  }
           { \hassubtype{\Gamma}{\intersect{T_t}{T_f}}{\tmap{T_t}{T_f}{\tbool}} }

\inferrule*[right=S-Map]
           { \hassubtype{\Gamma}{S}{T} }
           { \hassubtype{\Gamma}{\tmap{T_t}{T_f}{S}}{\tmap{T_t}{T_f}{T}} }

\end{mathpar}

Type $\tbool$ is defined as $\union{\ttrue}{\tfalse}$.

\subsection{Sub-Typing Relation 2: Type Evaluation Approach}

\begin{mathpar}

\inferrule*[right=S-Refl]
           {  }
           { \hassubtype{\Gamma}{S}{S} }

\inferrule*[right=S-Trans]
           { \hassubtype{\Gamma}{S}{U} \\ \hassubtype{\Gamma}{U}{T} }
           { \hassubtype{\Gamma}{S}{T} }

           \inferrule*[right=S-Top]
           {  }
           { \hassubtype{\Gamma}{S}{\ttop} }

\inferrule*[right=S-TVarSub]
           { \subtypeincontext{X}{T}{\Gamma} }
           { \hassubtype{\Gamma}{X}{T} }

\inferrule*[right=S-TVarSup]
           { \supertypeincontext{X}{T}{\Gamma} }
           { \hassubtype{\Gamma}{T}{X} }

\inferrule*[right=S-Arrow]
           { \hassubtype{\Gamma}{T_1}{S_1} \\ \hassubtype{\Gamma}{S_2}{T_2} }
           { \hassubtype{\Gamma}{(\fun{S_1}{S_2})}{(\fun{T_1}{T_2})} }

\inferrule*[right=S-All]
           { \hassubtype{\extendwsubtype{\Gamma}{X}{U}}{S}{T}}
           { \hassubtype{\Gamma}{(\tfun{X}{U}{S})}{(\tfun{X}{U}{T})} }

\inferrule*[right=S-UnionL]
           { \hassubtype{\Gamma}{T}{L} }
           { \hassubtype{\Gamma}{T}{\tbool} }

\inferrule*[right=S-UnionR]
           { \hassubtype{\Gamma}{T}{R} }
           { \hassubtype{\Gamma}{T}{\union{L}{R}} }

\inferrule*[right=S-UnionM]
           { \hassubtype{\Gamma}{L}{T} \\ \hassubtype{\Gamma}{R}{T} }
           { \hassubtype{\Gamma}{\union{L}{R}}{T} }

\inferrule*[right=S-TEvalRead]
           { \tevalr{\Gamma}{A}{A_2} \\ \hassubtype{\Gamma}{A_2}{B} }
           { \hassubtype{\Gamma}{A}{B} }

\inferrule*[right=S-TEvalWrite]
           { \tevalw{\Gamma}{B}{B_2} \\ \hassubtype{\Gamma}{A}{B_2} }
           { \hassubtype{\Gamma}{A}{B} }

\end{mathpar}

\subsection{Type Evaluation}

\begin{mathpar}
          
\inferrule*[right=TE-MapTrue]
           {  }
           { \teval{\Gamma}{\tmap{T_t}{T_f}{\ttrue}}{T_t} }

\inferrule*[right=TE-MapFalse]
           {  }
           { \teval{\Gamma}{\tmap{T_t}{T_f}{\tfalse}}{T_f} }

\inferrule*[right=TE-MapBoolRead]
           {  }
           { \tevalr{\Gamma}{\tmap{T_t}{T_f}{\tbool}}{\union{T_t}{T_f}} }

\inferrule*[right=TE-MapBoolWrite]
           {  }
           { \tevalw{\Gamma}{\tmap{T_t}{T_f}{\tbool}}{\intersect{T_t}{T_f}} }

\inferrule*[right=TE-MapRead]
           { \hassubtype{\Gamma}{A}{B} }
           { \tevalr{\Gamma}{\tmap{T_t}{T_f}{A}}{\tmap{T_t}{T_f}{B}} }

\inferrule*[right=TE-MapWrite]
           { \hassubtype{\Gamma}{A}{B} }
           { \tevalw{\Gamma}{\tmap{T_t}{T_f}{B}}{\tmap{T_t}{T_f}{A}} }

\end{mathpar}

\subsection{Sub-Typing Relation 3: Type Evaluation Approach (Algorithmic)}

\begin{mathpar}

\inferrule*[right=SA-Top]
           {  }
           { \hassubtype{\Gamma}{S}{\ttop} }

\inferrule*[right=SA-ReflTrue]
           {  }
           { \hassubtype{\Gamma}{\ttrue}{\ttrue} }

\inferrule*[right=SA-ReflFalse]
           {  }
           { \hassubtype{\Gamma}{\tfalse}{\tfalse} }

\inferrule*[right=SA-ReflTVar]
           { \subtypeincontext{X}{T}{\Gamma} }
           { \hassubtype{\Gamma}{X}{X} }

\inferrule*[right=SA-TrueSubX]
           { \subtypeincontext{True}{X}{\Gamma} \\ \subtypeincontext{X}{T}{\Gamma} }
           { \hassubtype{\Gamma}{True}{X} }

\inferrule*[right=SA-FalseSubX]
           { \subtypeincontext{False}{X}{\Gamma} \\ \subtypeincontext{X}{T}{\Gamma} }
           { \hassubtype{\Gamma}{False}{X} }

\inferrule*[right=SA-ReflMap]
           {  }
           { \hassubtype{\Gamma}{\tmap{T_t}{T_f}{T}}{\tmap{T_t}{T_f}{T}} }

\inferrule*[right=SA-TransTVar]
           { \subtypeincontext{X}{S}{\Gamma} \\ \hassubtype{\Gamma}{S}{T} }
           { \hassubtype{\Gamma}{X}{T} }

\inferrule*[right=SA-Arrow]
           { \hassubtype{\Gamma}{T_1}{S_1} \\ \hassubtype{\Gamma}{S_2}{T_2} }
           { \hassubtype{\Gamma}{(\fun{S_1}{S_2})}{(\fun{T_1}{T_2})} }

\inferrule*[right=SA-All]
           { \hassubtype{\Gamma}{U_1}{U_2} \\ \hassubtype{\extendwsubtype{\Gamma}{X}{U_2}}{S}{T}}
           { \hassubtype{\Gamma}{(\tfun{X}{U_1}{S})}{(\tfun{X}{U_2}{T})} }

\inferrule*[right=SA-UnionL]
           { \hassubtype{\Gamma}{T}{L} }
           { \hassubtype{\Gamma}{T}{\union{L}{R}} }

\inferrule*[right=SA-UnionR]
           { \hassubtype{\Gamma}{T}{R} }
           { \hassubtype{\Gamma}{T}{\union{L}{R}} }

\inferrule*[right=SA-UnionM]
           { \hassubtype{\Gamma}{L}{T} \\ \hassubtype{\Gamma}{R}{T} }
           { \hassubtype{\Gamma}{\union{L}{R}}{T} }

\inferrule*[right=SA-TEvalRead]
           { \tevalr{\Gamma}{\tmap{T_t}{T_f}{T}}{X} \\ \hassubtype{\Gamma}{X}{B} }
           { \hassubtype{\Gamma}{\tmap{T_t}{T_f}{T}}{B} }

\inferrule*[right=SA-TEvalWrite]
           { \tevalw{\Gamma}{\tmap{T_t}{T_f}{T}}{X} \\ \hassubtype{\Gamma}{A}{X} }
           { \hassubtype{\Gamma}{A}{\tmap{T_t}{T_f}{T}} }

\end{mathpar}

\subsection{Type Evaluation}

\begin{mathpar}
          
\inferrule*[right=TE-MapTrue]
           {  }
           { \teval{\Gamma}{\tmap{T_t}{T_f}{\ttrue}}{T_t} }

\inferrule*[right=TE-MapFalse]
           {  }
           { \teval{\Gamma}{\tmap{T_t}{T_f}{\tfalse}}{T_f} }

\inferrule*[right=TE-MapBoolRead]
           {  }
           { \tevalr{\Gamma}{\tmap{T_t}{T_f}{\tbool}}{\union{T_t}{T_f}} }

\inferrule*[right=TE-MapRead]
           { \hassubtype{\Gamma}{A}{B} \\ \tevalr{\Gamma}{\tmap{T_t}{T_f}{B}}{X} }
           { \tevalr{\Gamma}{\tmap{T_t}{T_f}{A}}{X} }

\inferrule*[right=TE-MapWrite]
           { \hassubtype{\Gamma}{B}{A} \\ \tevalw{\Gamma}{\tmap{T_t}{T_f}{B}}{X} }
           { \tevalw{\Gamma}{\tmap{T_t}{T_f}{A}}{X} }

\end{mathpar}

\subsection{Bidirectional Typing}

\begin{mathpar}

\inferrule*[right=BT-Ann]
           { \checks{\Gamma}{t}{T} }
           { \infers{\Gamma}{\ann{t}{T}}{T} }

\inferrule*[right=BT-True]
           {  }
           { \infers{\Gamma}{\true}{\ttrue} }

\inferrule*[right=BT-False]
           {  }
           { \infers{\Gamma}{\false}{\tfalse} }

\inferrule*[right=BT-Var]
           { \typeincontext{x}{T}{\Gamma} }
           { \infers{\Gamma}{x}{T} }

\inferrule*[right=BT-Abs]
           { \checks{\extendwtype{\Gamma}{x}{A}}{t}{B} }
           { \checks{\Gamma}{\babs{x}{t}}{\fun{A}{B}} }

\inferrule*[right=BT-App]
           { \infers{\Gamma}{t_1}{\fun{A}{B}} \\ \checks{\Gamma}{t_2}{A} }
           { \infers{\Gamma}{\app{t_1}{t_2}}{B} }

\inferrule*[right=BT-TAbs]
           { \infers{\extendwsubtype{\Gamma}{X}{U}}{t}{T} }
           { \infers{\Gamma}{(\tabs{X}{U}{t})}{(\tfun{X}{U}{T})} }

\inferrule*[right=BT-TApp]
           { \infers{\Gamma}{t}{\tfun{X}{U}{T}} \\ \hassubtype{\Gamma}{S}{U} }
           { \infers{\Gamma}{\tapp{t}{S}}{\subst{T}{X}{S}} }

\inferrule*[right=BT-Sub]
           { \infers{\Gamma}{t}{S} \\ \hassubtype{\Gamma}{S}{T} }
           { \checks{\Gamma}{t}{T} }

\inferrule*[right=BT-Eq]
           {  }
           { \infers{\Gamma}{\eq{t_1}{t_2}}{\tbool} }

\inferrule*[right=T-If]
           { \checks{\Gamma}{t_1}{\tbool} \\ \infers{\Gamma}{t_2}{A} \\ \infers{\Gamma}{t_3}{B} }
           { \infers{\Gamma}{\ite{t_1}{t_2}{t_3}}{\union{A}{B}} }

\inferrule*[right=T-IfTrue]
           { \typeincontext{x}{X}{\Gamma} \\ \subtypeincontext{X}{T}{\Gamma} \\ \infers{\extendwsubtype{\tsubst{\Gamma}{x}{\true}}{\ttrue}{X}}{t_2}{A} \\ \infers{\extendwsubtype{\tsubst{\Gamma}{x}{\false}}{\tfalse}{X}}{t_3}{B} }
           { \infers{\Gamma}{\ite{\eq{x}{\true}}{t_2}{t_3}}{\union{A}{B}} }


\end{mathpar}

\subsection{Term Evaluation}

\begin{mathpar}

\inferrule*[right=E-AppAbs]
           {  }
           { \eval{\app{(\babs{x}{t})}{v_2}}{\subst{t}{x}{v_2}} }

\inferrule*[right=E-App1]
           { \eval{t_1}{t_1'} }
           { \eval{\app{t_1}{t_2}}{\app{t_1'}{t_2}} }

\inferrule*[right=E-App2]
           { \eval{t_2}{t_2'} }
           { \eval{\app{v_1}{t_2}}{\app{v_1}{t_2'}} }

\inferrule*[right=E-TappTabs]
           {  }
           { \eval{\tapp{(\tabs{X}{U}{t})}{T}}{\tsubst{t}{X}{T}} }

\inferrule*[right=E-Tapp]
           { \eval{t}{t'} }
           { \eval{\tapp{t}{T}}{\tapp{t'}{T}} }

\inferrule*[right=E-IfTrue]
           {  }
           { \eval{\ite{\true}{t_1}{t_2}}{t_1} }

\inferrule*[right=E-IfFalse]
           {  }
           { \eval{\ite{\false}{t_1}{t_2}}{t_2} }

\inferrule*[right=E-If]
           { \eval{t}{t'} }
           { \eval{\ite{t}{t_1}{t_2}}{\ite{t'}{t_1}{t_2}} }

\inferrule*[right=E-EqL]
           { \eval{t_1}{t_1'} }
           { \eval{\eq{t_1}{t_2}}{\eq{t_1'}{t_2}} }

\inferrule*[right=E-EqR]
           { \eval{t_2}{t_2'} }
           { \eval{\eq{v_1}{t_2}}{\eq{v_1}{t_2'}} }

\inferrule*[right=E-EqTrue]
           { v_1 = v_2 }
           { \eval{\eq{v_1}{v_2}}{\true} }

\inferrule*[right=E-EqFalse]
           { v_1 \neq v_2 }
           { \eval{\eq{v_1}{v_2}}{\false} }



\end{mathpar}
