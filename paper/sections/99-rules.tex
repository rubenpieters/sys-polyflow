\section{Rules}

\subsection{Typing Relation}

\begin{mathpar}

\inferrule*[right=T-True]
           {  }
           { \hastype{\Gamma}{\true}{\ttrue} }

\inferrule*[right=T-False]
           {  }
           { \hastype{\Gamma}{\false}{\tfalse} }

\inferrule*[right=T-Var]
           { \typeincontext{x}{T}{\Gamma} }
           { \hastype{\Gamma}{x}{T} }

\inferrule*[right=T-Abs]
           { \hastype{\extendwtype{\Gamma}{x}{A}}{t}{B} }
           { \hastype{\Gamma}{(\abs{x}{A}{t}{B})}{(\fun{A}{B})} }

\inferrule*[right=T-App]
           { \hastype{\Gamma}{t_1}{\fun{A}{B}} \\ \hastype{\Gamma}{t_2}{A} }
           { \hastype{\Gamma}{\app{t_1}{t_2}}{B} }

\inferrule*[right=T-TAbs]
           { \hastype{\extendwsubtype{\Gamma}{X}{U}}{t}{T} }
           { \hastype{\Gamma}{(\tabs{X}{U}{t})}{(\tfun{X}{U}{T})} }

\inferrule*[right=T-TApp]
           { \hastype{\Gamma}{t}{\tfun{X}{U}{T}} \\ \hassubtype{\Gamma}{S}{U} }
           { \hastype{\Gamma}{\tapp{t}{S}}{\subst{T}{X}{S}} }

\inferrule*[right=T-Sub]
           { \hastype{\Gamma}{t}{S} \\ \hassubtype{\Gamma}{S}{T} }
           { \hastype{\Gamma}{t}{T} }

\inferrule*[right=T-Eq]
           { \hastype{\Gamma}{t_1}{T} \\ \hastype{\Gamma}{t_2}{T} }
           { \hastype{\Gamma}{\eq{t_1}{t_2}}{\tbool} }

\inferrule*[right=T-If]
           { \hastype{\Gamma}{t_1}{\tbool} \\ \hastype{\Gamma}{t_2}{A} \\ \hastype{\Gamma}{t_3}{B} }
           { \hastype{\Gamma}{\ite{t_1}{t_2}{t_3}}{\union{A}{B}} }

% NOTE: in flow-typing system with intersection/negation types, it becomes Gamma [ x |-> Gamma(x) & T ] in 'x is T' branch, and Gamma [ x |-> Gamma(x) & not(T) ] in else branch
% for example in Sound and Complete Flow Typing with Unions, Intersections and Negations           

\inferrule*[right=T-IfTrue]
           { \hastype{\Gamma}{x}{X} \\ \hastype{\extendwsubtype{\tsubst{\Gamma}{x}{\true}}{\ttrue}{X}}{t_2}{A} \\ \hastype{\extendwsubtype{\tsubst{\Gamma}{x}{\tfalse}}{\tfalse}{X}}{t_3}{B} }
           { \hastype{\Gamma}{\ite{\eq{x}{\true}}{t_2}{t_3}}{\union{A}{B}} }

\end{mathpar}

\subsection{Sub-Typing Relation 1: Sub-Typing Approach}

\begin{mathpar}

\inferrule*[right=S-Refl]
           {  }
           { \hassubtype{\Gamma}{S}{S} }

\inferrule*[right=S-Trans]
           { \hassubtype{\Gamma}{S}{U} \\ \hassubtype{\Gamma}{U}{T} }
           { \hassubtype{\Gamma}{S}{T} }

           \inferrule*[right=S-Top]
           {  }
           { \hassubtype{\Gamma}{S}{\ttop} }

\inferrule*[right=S-TVarSub]
           { \subtypeincontext{X}{T}{\Gamma} }
           { \hassubtype{\Gamma}{X}{T} }

\inferrule*[right=S-TVarSup]
           { \supertypeincontext{X}{T}{\Gamma} }
           { \hassubtype{\Gamma}{T}{X} }

\inferrule*[right=S-Arrow]
           { \hassubtype{\Gamma}{T_1}{S_1} \\ \hassubtype{\Gamma}{S_2}{T_2} }
           { \hassubtype{\Gamma}{(\fun{S_1}{S_2})}{(\fun{T_1}{T_2})} }

\inferrule*[right=S-All]
           { \hassubtype{\extendwsubtype{\Gamma}{X}{U}}{S}{T}}
           { \hassubtype{\Gamma}{(\tfun{X}{U}{S})}{(\tfun{X}{U}{T})} }

\inferrule*[right=S-UnionL]
           { \hassubtype{\Gamma}{T}{L} }
           { \hassubtype{\Gamma}{T}{\union{L}{R}} }

\inferrule*[right=S-UnionR]
           { \hassubtype{\Gamma}{T}{R} }
           { \hassubtype{\Gamma}{T}{\union{L}{R}} }

\inferrule*[right=S-UnionM]
           { \hassubtype{\Gamma}{L}{T} \\ \hassubtype{\Gamma}{R}{T} }
           { \hassubtype{\Gamma}{\union{L}{R}}{T} }

\inferrule*[right=S-MapTrueL]
           {  }
           { \hassubtype{\Gamma}{\tmap{T_t}{T_f}{\ttrue}}{T_t} }

\inferrule*[right=S-MapTrueR]
           {  }
           { \hassubtype{\Gamma}{T_t}{\tmap{T_t}{T_f}{\ttrue}} }

\inferrule*[right=S-MapFalseL]
           {  }
           { \hassubtype{\Gamma}{\tmap{T_t}{T_f}{\tfalse}}{T_f} }

\inferrule*[right=S-MapFalseR]
           {  }
           { \hassubtype{\Gamma}{T_f}{\tmap{T_t}{T_f}{\tfalse}} }

\inferrule*[right=S-MapBoolL]
           {  }
           { \hassubtype{\Gamma}{\tmap{T_t}{T_f}{\tbool}}{\union{T_t}{T_f}} }

\inferrule*[right=(S-MapBoolR)]
           {  }
           { \hassubtype{\Gamma}{\intersect{T_t}{T_f}}{\tmap{T_t}{T_f}{\tbool}} }

\inferrule*[right=S-Map]
           { \hassubtype{\Gamma}{S}{T} }
           { \hassubtype{\Gamma}{\tmap{T_t}{T_f}{S}}{\tmap{T_t}{T_f}{T}} }

\end{mathpar}

Type $\tbool$ is defined as $\union{\ttrue}{\tfalse}$.

\subsection{Sub-Typing Relation 2: Type Evaluation Approach}

\begin{mathpar}

\inferrule*[right=S-Refl]
           {  }
           { \hassubtype{\Gamma}{S}{S} }

\inferrule*[right=S-Trans]
           { \hassubtype{\Gamma}{S}{U} \\ \hassubtype{\Gamma}{U}{T} }
           { \hassubtype{\Gamma}{S}{T} }

           \inferrule*[right=S-Top]
           {  }
           { \hassubtype{\Gamma}{S}{\ttop} }

\inferrule*[right=S-TVarSub]
           { \subtypeincontext{X}{T}{\Gamma} }
           { \hassubtype{\Gamma}{X}{T} }

\inferrule*[right=S-TVarSup]
           { \supertypeincontext{X}{T}{\Gamma} }
           { \hassubtype{\Gamma}{T}{X} }

\inferrule*[right=S-Arrow]
           { \hassubtype{\Gamma}{T_1}{S_1} \\ \hassubtype{\Gamma}{S_2}{T_2} }
           { \hassubtype{\Gamma}{(\fun{S_1}{S_2})}{(\fun{T_1}{T_2})} }

\inferrule*[right=S-All]
           { \hassubtype{\extendwsubtype{\Gamma}{X}{U}}{S}{T}}
           { \hassubtype{\Gamma}{(\tfun{X}{U}{S})}{(\tfun{X}{U}{T})} }

\inferrule*[right=S-UnionL]
           { \hassubtype{\Gamma}{T}{L} }
           { \hassubtype{\Gamma}{T}{\tbool} }

\inferrule*[right=S-UnionR]
           { \hassubtype{\Gamma}{T}{R} }
           { \hassubtype{\Gamma}{T}{\union{L}{R}} }

\inferrule*[right=S-UnionM]
           { \hassubtype{\Gamma}{L}{T} \\ \hassubtype{\Gamma}{R}{T} }
           { \hassubtype{\Gamma}{\union{L}{R}}{T} }

\inferrule*[right=S-TEvalRead]
           { \tevalr{\Gamma}{A}{A_2} \\ \hassubtype{\Gamma}{A_2}{B} }
           { \hassubtype{\Gamma}{A}{B} }

\inferrule*[right=S-TEvalWrite]
           { \tevalw{\Gamma}{B}{B_2} \\ \hassubtype{\Gamma}{A}{B_2} }
           { \hassubtype{\Gamma}{A}{B} }

\end{mathpar}

\subsection{Type Evaluation}

\begin{mathpar}
          
\inferrule*[right=TE-MapTrue]
           {  }
           { \teval{\Gamma}{\tmap{T_t}{T_f}{\ttrue}}{T_t} }

\inferrule*[right=TE-MapFalse]
           {  }
           { \teval{\Gamma}{\tmap{T_t}{T_f}{\tfalse}}{T_f} }

\inferrule*[right=TE-MapBoolRead]
           {  }
           { \tevalr{\Gamma}{\tmap{T_t}{T_f}{\tbool}}{\union{T_t}{T_f}} }

\inferrule*[right=TE-MapBoolWrite]
           {  }
           { \tevalw{\Gamma}{\tmap{T_t}{T_f}{\tbool}}{\intersect{T_t}{T_f}} }

\inferrule*[right=TE-MapRead]
           { \hassubtype{\Gamma}{A}{B} }
           { \tevalr{\Gamma}{\tmap{T_t}{T_f}{A}}{\tmap{T_t}{T_f}{B}} }

\inferrule*[right=TE-MapWrite]
           { \hassubtype{\Gamma}{A}{B} }
           { \tevalw{\Gamma}{\tmap{T_t}{T_f}{B}}{\tmap{T_t}{T_f}{A}} }

\end{mathpar}

\subsection{Grammar}

- Types
\\$A,B,L,R,T_t,T_f,T,U ::= \ttop\alt\fun{A}{B}\alt\union{L}{R}\alt\tfun{X}{U}{T}\alt\ttrue\alt\tfalse\alt\tmap{T_t}{T_f}{T}\alt{}X$
\\- Terms
\\$t ::= \true\alt\false\alt\babs{x}{t}\alt\app{t_1}{t_2}\alt\tabs{X}{U}{t}\alt\tapp{t}{T}\alt\ite{t}{t_1}{t_2}\alt\eq{t_1}{t_2}\alt\tand{t_1}{t_2}\alt\ann{t}{T}$
\\- Values
\\$v ::= \true\alt\false\alt\babs{x}{t}\alt\tabs{X}{U}{t}$
\\- Context
\\$\Gamma ::= \cdot\alt\Gamma{}, x : A\alt\Gamma{}, S < X <: T$
\\- Condition Context
\\$\Delta ::= \emptyset\alt\Delta{}, S < X$
\\Invariants context: each variabele has max 1 type, each type-variable has max 1 upper bound and max 1 lower bound (what in situation with more literals and false-branches ? formulate invariant more generic ?), if type variable has a lower bound it has an upper bound, no cycles allowed
\\- Type Evaluation Modes
\\$\texttt{m} ::= r\alt{}w$

\subsection{Sub-Typing Relation 3: Type Evaluation Approach (Algorithmic)}

\begin{mathpar}

\inferrule*[right=SA-Top]
           {  }
           { \hassubtype{\Gamma}{S}{\ttop} }

\inferrule*[right=SA-ReflTrue]
           {  }
           { \hassubtype{\Gamma}{\ttrue}{\ttrue} }

\inferrule*[right=SA-ReflFalse]
           {  }
           { \hassubtype{\Gamma}{\tfalse}{\tfalse} }

\inferrule*[right=SA-ReflTVar]
           { \subtypeincontext{X}{T}{\Gamma} }
           { \hassubtype{\Gamma}{X}{X} }

%\inferrule*[right=SA-TrueSubX]
%           { \subtypeincontext{True}{X}{\Gamma} }
%           { \hassubtype{\Gamma}{True}{X} }

%\inferrule*[right=SA-FalseSubX]
%           { \subtypeincontext{False}{X}{\Gamma} }
%           { \hassubtype{\Gamma}{False}{X} }

\inferrule*[right=SA-ReflMap]
           {  }
           { \hassubtype{\Gamma}{\tmap{T_t}{T_f}{T}}{\tmap{T_t}{T_f}{T}} }

\inferrule*[right=SA-TransTVar]
           { \subtypeincontext{X}{S}{\Gamma} \\ \hassubtype{\Gamma}{S}{T} }
           { \hassubtype{\Gamma}{X}{T} }

\inferrule*[right=SA-Arrow]
           { \hassubtype{\Gamma}{T_1}{S_1} \\ \hassubtype{\Gamma}{S_2}{T_2} }
           { \hassubtype{\Gamma}{(\fun{S_1}{S_2})}{(\fun{T_1}{T_2})} }

\inferrule*[right=SA-All]
           { \hassubtype{\extendwsubtype{\Gamma}{X}{U}}{S}{T}}
           { \hassubtype{\Gamma}{(\tfun{X}{U}{S})}{(\tfun{X}{U}{T})} }

\inferrule*[right=SA-UnionL]
           { \hassubtype{\Gamma}{T}{L} }
           { \hassubtype{\Gamma}{T}{\union{L}{R}} }

\inferrule*[right=SA-UnionR]
           { \hassubtype{\Gamma}{T}{R} }
           { \hassubtype{\Gamma}{T}{\union{L}{R}} }

\inferrule*[right=SA-UnionM]
           { \hassubtype{\Gamma}{L}{T} \\ \hassubtype{\Gamma}{R}{T} }
           { \hassubtype{\Gamma}{\union{L}{R}}{T} }

\inferrule*[right=SA-TEvalRead]
           { \tevalrx{\Gamma}{\tmap{T_t}{T_f}{T}}{X} \\ \hassubtype{\Gamma}{X}{A} }
           { \hassubtype{\Gamma}{\tmap{T_t}{T_f}{T}}{A} }

\inferrule*[right=SA-TEvalWrite]
           { \tevalwx{\Gamma}{\tmap{T_t}{T_f}{T}}{X} \\ \hassubtype{\Gamma}{A}{X} }
           { \hassubtype{\Gamma}{A}{\tmap{T_t}{T_f}{T}} }

\end{mathpar}

\subsection{Type Evaluation}

\begin{mathpar}
          
\inferrule*[right=TE-MapTrue]
           {  }
           { \teval{\Gamma}{\tmap{T_t}{T_f}{\ttrue}}{T_t} }

\inferrule*[right=TE-MapFalse]
           {  }
           { \teval{\Gamma}{\tmap{T_t}{T_f}{\tfalse}}{T_f} }

\inferrule*[right=TE-MapUnion]
           {  }
           { \tevalr{\Gamma}{ \tmap{T_t}{T_f}{\union{L}{R}} }{ \union{ \tmap{T_t}{T_f}{L} }{ \tmap{T_t}{T_f}{R} } } }

\inferrule*[right=TE-TVarRead]
           { S < \subtypeincontext{X}{T}{\Gamma} }
           { \tevalr{\Gamma}{ X }{ T } }

\inferrule*[right=TE-TVarWrite]
           %{ \subtypeincontext{U}{X}{\Gamma} }
           { \{ U \} < X <: T \in \Gamma }
           { \tevalw{\Gamma}{ X }{ U } }

\inferrule*[right=TE-Map]
           { \teval{\Gamma}{ T }{ T' } }
           { \teval{\Gamma}{ \tmap{T_t}{T_f}{T} }{ \tmap{T_t}{T_f}{T'} } }

\inferrule*[right=TE-UnionL]
           { \teval{\Gamma}{ L }{ L' } }
           { \teval{\Gamma}{ \union{L}{R} }{ \union{L'}{R} } }

\inferrule*[right=TE-UnionR]
           { \teval{\Gamma}{ R }{ R' } }
           { \teval{\Gamma}{ \union{L}{R} }{ \union{L}{R'} } }

\end{mathpar}

The relation $\tevalx{\Gamma}{T_1}{T_2}$ is the transitive closure of $\teval{\Gamma}{T_1}{T_2}$.

\subsection{Bidirectional Typing}

\begin{mathpar}

\inferrule*[right=BT-Ann]
           { \checks{\Gamma}{t}{T} }
           { \infers{\Gamma}{ \ann{t}{T} \  }{ T } }

\inferrule*[right=BT-True]
           {  }
           { \infers{\Gamma}{\true}{\ttrue} }

\inferrule*[right=BT-False]
           {  }
           { \infers{\Gamma}{\false}{\tfalse} }

\inferrule*[right=BT-Var]
           { \typeincontext{x}{T}{\Gamma} }
           { \infers{\Gamma}{x}{T} }

\inferrule*[right=BT-App]
           { \infers{\Gamma}{t_1}{\fun{A}{B}} \\ \checks{\Gamma}{t_2}{A} }
           { \infers{\Gamma}{\app{t_1}{t_2}}{B} }

\inferrule*[right=BT-TAbs]
           { \infers{\extendwsubtype{\Gamma}{X}{U}}{t}{T} }
           { \infers{\Gamma}{(\tabs{X}{U}{t})}{(\tfun{X}{U}{T})} }

\inferrule*[right=BT-TApp]
           { \infers{\Gamma}{t}{\tfun{X}{U}{T}} \\ \hassubtype{\Gamma}{S}{U} }
           { \infers{\Gamma}{\tapp{t}{S}}{\subst{T}{X}{S}} }

\inferrule*[right=BT-Eq]
           { \infers{\Gamma}{t_1}{T_1} \\ \infers{\Gamma}{t_2}{T_2} }
           { \infers{\Gamma}{\eq{t_1}{t_2}}{\tbool} }

\inferrule*[right=BT-And]
           { \infers{\Gamma}{t_1}{T_1} \\ \infers{\Gamma}{t_2}{T_2} \\ \hassubtype{\Gamma}{T_1}{Bool} \\ \hassubtype{\Gamma}{T_2}{Bool} }
           { \infers{\Gamma}{\tand{t_1}{t_2}}{\tbool} }

\inferrule*[right=BT-If]
           { \checks{\Gamma}{t_1}{\tbool}  \\ \ci{\Gamma}{t_1}{\Delta} \\ \infers{\coi{\Gamma}{\Delta}}{t_2}{A} \\ \infers{\coi{\Gamma}{ (\Gamma \vdash\con{\Delta}}) }{t_3}{B} }
           { \infers{\Gamma}{\ite{t_1}{t_2}{t_3}}{\union{A}{B}} }

\end{mathpar}

\begin{mathpar}

\inferrule*[right=BT-Abs]
           { \checks{\extendwtype{\Gamma}{x}{A}}{t}{B} }
           { \checks{\Gamma}{\babs{x}{t}}{\fun{A}{B}} }

\inferrule*[right=BT-Sub]
           { \infers{\Gamma}{t}{S} \\ \hassubtype{\Gamma}{S}{T} }
           { \checks{\Gamma}{t}{T} }

\end{mathpar}

\subsection{Condition Information}

\begin{mathpar}

\inferrule*[right=CI-True]
           {  }
           { \ci{\Gamma}{ \true }{ \emptyset } }

\inferrule*[right=CI-False]
           {  }
           { \ci{\Gamma}{ \false }{ \emptyset } }

\inferrule*[right=CI-EqTrueR]
           { \typeincontext{x}{X}{\Delta} \\ \subtypeincontext{X}{T}{\Gamma} \\ \hassubtype{\Gamma}{T}{Bool} }
           { \ci{\Gamma}{ \eq{x}{\true} }{ \emptyset , \{ \ttrue \} < X } }

\inferrule*[right=CI-EqTrueL]
           { \typeincontext{x}{X}{\Gamma} \\ \subtypeincontext{X}{T}{\Gamma} \\ \hassubtype{\Gamma}{T}{Bool} }
           { \ci{\Gamma}{ \eq{\true}{x} }{ \emptyset , \{ \ttrue \} < X } }

\inferrule*[right=CI-EqFalseR]
           { \typeincontext{x}{X}{\Gamma} \\ \subtypeincontext{X}{T}{\Gamma} \\ \hassubtype{\Gamma}{T}{Bool} }
           { \ci{\Gamma}{ \eq{x}{\false} }{ \emptyset , \{ \tfalse \} < X } }

\inferrule*[right=CI-EqFalseL]
           { \typeincontext{x}{X}{\Gamma} \\ \subtypeincontext{X}{T}{\Gamma} \\ \hassubtype{\Gamma}{T}{Bool} }
           { \ci{\Gamma}{ \eq{\false}{x} }{ \emptyset , \{ \tfalse \} < X } }

\inferrule*[right=CI-EqAnd]
           { \ci{\Gamma}{t_1}{\Delta_1} \\ \ci{\Gamma}{t_2}{\Delta_2} }
           { \ci{\Gamma}{ \tand{t_1}{t_2} }{ \coi{\Delta_1}{\Delta_2} } }

%\inferrule*[right=CI-EqOr]
%           { \ci{\Gamma}{t_1}{\Gamma_1} \\ \ci{\Gamma}{t_2}{\Gamma_2} }
%           { \ci{\Gamma}{ \tor{t_1}{t_2} }{ \cou{\Gamma_1}{ (\cou{\Gamma_2}{\emptyset}) } } }

\inferrule*[right=CI-EqOtherwise]
           { \mathit{otherwise} }
           { \ci{\Delta}{ t }{ \emptyset } }

\end{mathpar}

%\begin{mathpar}
%
%\inferrule*[right=CI-EqTrueR]
%           { \typeincontext{x}{X}{\Gamma} \\ \subtypeincontext{X}{T}{\Gamma} \\ \hassubtype{\Gamma}{T}{Bool} }
%           { \ci{\Gamma}{ \eq{x}{\true} }{ \Gamma \rightarrow \Gamma[x] = \Gamma[x] \cap \{ \true \} } }
%
%\end{mathpar}

\subsection{Context Operations}

\begin{mathpar}

\inferrule*[right=CO-IntersectEmpty]
           {  }
           { \coia{\Gamma}{\emptyset}{\Gamma} }

\inferrule*[right=CO-Intersect]
           { S_1 < X <: T \in \Gamma }
           { \coia{\Gamma}{ (\Delta , S_2 < X) }{ \Gamma[X \mapsto (S_1 \cap S_2) < X <: T] \sqcap \Delta } }

\inferrule*[right=CO-SmallEmpty]
           {  }
           { \coia{\Delta}{\emptyset}{\Delta} }

\inferrule*[right=CO-SmallIntersect]
           { S_1 < X \in \Delta_1 }
           { \coia{\Delta_1}{ (\Delta_2 , S_2 < X) }{ \Delta_1[X \mapsto (S_1 \cap S_2) < X] \sqcap \Delta_2 } }

\inferrule*[right=CO-SmallAdd]
           { S_1 < X \notin \Delta_1 }
           { \coia{\Delta_1}{ (\Delta_2 , S_2 < X) }{ \Delta_1[X \mapsto S_2 < X] \sqcap \Delta_2 } }


%\inferrule*[right=CO-Intersect1]
%           {  }
%           { \coia{\emptyset}{\Gamma}{\Gamma} }
%
%\inferrule*[right=CO-Intersect2]
%           { (S_2 < X) \in \Gamma \\ \Delta = \Gamma \setminus (S_2 < X) }
%           { \coia{(\emptyset , S_1 < X)}{\Gamma}{\Delta , (S_1 \cap S_2) < X} }
%
%\inferrule*[right=CO-Intersect3]
%           { \Gamma\,\cancel{\vdash}\,x \in S_2 }
%           { \coia{(\emptyset , X \in S_1)}{\Gamma}{\Gamma , X \in S_1} }
%
%\inferrule*[right=CO-Intersect4]
%           { \Delta \neq \emptyset }
%           { \coia{\Delta , X \in S}{\Gamma}{ \coi{\Delta}{(\coi{(\emptyset , X \in S)}{\Gamma)}} } }
        
\end{mathpar}

%\begin{mathpar}
%
%\inferrule*[right=CO-Union1]
%           {  }
%           { \coua{\emptyset}{\Gamma}{\Gamma} }
%
%\inferrule*[right=CO-Union2]
%           { (X \in S_2) \in \Gamma \\ \Delta = \Gamma \setminus (X \in S_2) }
%           { \coua{(\emptyset , X \in S_1)}{\Gamma}{\Delta , X \in S_1 \cup S_2} }
%
%\inferrule*[right=CO-Union3]
%           { \Gamma\,\cancel{\vdash}\,x \in S_2 }
%           { \coua{(\emptyset , X \in S_1)}{\Gamma}{\Gamma} }
%
%\inferrule*[right=CO-Union4]
%           { \Delta \neq \emptyset }
%           { \coua{\Delta , X \in S}{\Gamma}{ \cou{\Delta}{(\cou{(\emptyset , X \in S)}{\Gamma)}} } }
%        
%\end{mathpar}

\begin{mathpar}

\inferrule*[right=CO-InverseEmpty]
           {  }
           { \Gamma \vdash \cona{\emptyset}{\emptyset} }

\inferrule*[right=CO-InverseCons]
           { U < X <: T \in \Gamma \\ U \setminus S_1 = S_2 }
           { \Gamma \vdash \cona{(\Delta , S < X)}{ \con{\Delta}, S_2 < X } }
        
\end{mathpar}

\subsection{Term Evaluation}

\begin{mathpar}

\inferrule*[right=E-AppAbs]
           {  }
           { \eval{\app{(\babs{x}{t})}{v_2}}{\subst{t}{x}{v_2}} }

\inferrule*[right=E-App1]
           { \eval{t_1}{t_1'} }
           { \eval{\app{t_1}{t_2}}{\app{t_1'}{t_2}} }

\inferrule*[right=E-App2]
           { \eval{t_2}{t_2'} }
           { \eval{\app{v_1}{t_2}}{\app{v_1}{t_2'}} }

\inferrule*[right=E-TappTabs]
           {  }
           { \eval{\tapp{(\tabs{X}{U}{t})}{T}}{\tsubst{t}{X}{T}} }

\inferrule*[right=E-Tapp]
           { \eval{t}{t'} }
           { \eval{\tapp{t}{T}}{\tapp{t'}{T}} }

\inferrule*[right=E-IfTrue]
           {  }
           { \eval{\ite{\true}{t_1}{t_2}}{t_1} }

\inferrule*[right=E-IfFalse]
           {  }
           { \eval{\ite{\false}{t_1}{t_2}}{t_2} }

\inferrule*[right=E-If]
           { \eval{t}{t'} }
           { \eval{\ite{t}{t_1}{t_2}}{\ite{t'}{t_1}{t_2}} }

\inferrule*[right=E-EqL]
           { \eval{t_1}{t_1'} }
           { \eval{\eq{t_1}{t_2}}{\eq{t_1'}{t_2}} }

\inferrule*[right=E-EqR]
           { \eval{t_2}{t_2'} }
           { \eval{\eq{v_1}{t_2}}{\eq{v_1}{t_2'}} }

\inferrule*[right=E-EqTrue]
           { v_1 = v_2 }
           { \eval{\eq{v_1}{v_2}}{\true} }

\inferrule*[right=E-EqFalse]
           { v_1 \neq v_2 }
           { \eval{\eq{v_1}{v_2}}{\false} }

\inferrule*[right=E-AndL]
           { \eval{t_1}{t_1'} }
           { \eval{\tand{t_1}{t_2}}{\tand{t_1'}{t_2}} }

\inferrule*[right=E-AndR]
           { \eval{t_2}{t_2'} }
           { \eval{\tand{t_1}{t_2}}{\tand{t_1}{t_2'}} }

\inferrule*[right=E-AndFalseL]
           {  }
           { \eval{\tand{\false}{v_2}}{\false} }

\inferrule*[right=E-AndFalseR]
           {  }
           { \eval{\tand{v_1}{\false}}{\false} }

\inferrule*[right=E-AndTrue]
           {  }
           { \eval{\tand{\true}{\true}}{\true} }


%\inferrule*[right=E-OrL]
%           { \eval{t_1}{t_1'} }
%           { \eval{\tor{t_1}{t_2}}{\tor{t_1'}{t_2}} }
%
%\inferrule*[right=E-OrR]
%           { \eval{t_2}{t_2'} }
%           { \eval{\tor{t_1}{t_2}}{\tor{t_1}{t_2'}} }
%
%\inferrule*[right=E-Or]
%           { t_1\,\lor\,t_2 = r }
%           { \eval{\tor{t_1}{t_2}}{r} }

\end{mathpar}
